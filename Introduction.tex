Evolutionary Algorithms (\EAS{}) are one of the most widely used techniques to deal with complex optimization problems.
%
Several variants of these strategies have been devised~\cite{Talbi:09} and applied in many fields, such as in science, 
economic and engineering~\cite{chakraborty2008advances}.
%
Among them, Differential Evolution (\DE{}) \cite{storn1997differential} is one of the most effective strategies to deal
with continuous optimization.
%
In fact, it has been the winning strategy of several optimization competitions~\cite{das2011differential}.
%
Similarly to other \EAS{}, \DE{} is inspired by the natural evolution process and it involves the application of mutation, recombination and 
selection.
%
The main peculiarity of \DE{} is that it considers the differences among vectors, which are present in the population to explore the search space.
%
In this sense is similar to Nelder-Mead \cite{nelder1965simplex} and the 
Controlled Random Search (CRS) \cite{price1983global} optimizers.

In spite of the effectiveness of \DE{}, there exists several weaknesses that have been detected and partially
solved by extending the standard variant~\cite{das2011differential}.
%
Among them, the sensitivity to its parameters~\cite{zhang2009jade}, the appearance of stagnation due to the reduced exploration 
capabilities~\cite{sa2008exploration,lampinen2000stagnation} and premature convergence~\cite{zaharie2003control} are some of the most well-known
issues.
%
This last one issue is tackled in this paper.
%
Note that, attending to the proper design of population-based meta-heuristics~\cite{Talbi:09}, special attention must be
paid to attain a proper balance between exploration and exploitation.
%
A too large exploration degree prevents the proper intensification of the best located regions, usually resulting in a
too slow convergence.
%
Differently, an excessive exploitation degree provokes loss of diversity meaning that only a limited number of regions are sampled.

Since the appearance of \DE{}, some criticism appeared because of its incapability to maintain a large
enough diversity due to the use of a selection with high pressure~\cite{sa2008exploration}.
%
Thus, several extensions of \DE{} to deal with premature convergence have been devised 
such as parameter adaptation~\cite{zaharie2003control}, 
auto-enhanced population diversity~\cite{yang2015differential} and selection strategies 
with a lower selection pressure~\cite{sa2008exploration}.
%
Some of the last studies on design of population-based meta-heuristics~\cite{Crepinsek:13} show that 
properly balancing the exploration and intensification
is particularly useful for avoiding premature convergence.
%
Specifically, in the field of combinatorial optimization some novel replacement strategies that dynamically alter 
the balance between exploration and exploitation 
have appeared~\cite{segura2016novel}.
%
The main principle of such proposals is to use the stopping criterion and elapsed generations to bias the decisions 
taken by the optimizers with the aim of promoting exploration in the initial stages and exploitation in the last ones.
%
Probably their main weakness is that the time required to obtain high-quality solution increases.
%
Our novel proposal, which is called \DE{} with Enhanced Diversity Maintenance (\DEEDM{}), integrates a similar principle into \DE{}.
%
However, in order to avoid the excessive growth of computational requirements typical of diversity-based replacement strategies, 
the method was extended with the aim of inducing a larger degree of intensification.

The rest of the paper is organized as follows.
%
Some basic concepts of \DE{} and a review of works related to diversity within \DE{} are given in section~\ref{sec:Literature}.
%
Section~\ref{sec:trends} presents an analysis about the algorithms with best performance on the last continuous optimization 
contests held at the \textsc{ieee} Congress on Evolutionary Computation.
%
More emphasis is given on the variants based on \DE{}.
%
Our proposal is described in section~\ref{sec:Proposal}.
%
The experimental validation, which includes comparisons against state-of-the-art approaches, is shown in section~\ref{sec:Experimental}. 
%
Finally, our conclusions and some lines of future work are given in section~\ref{sec:Conclusion}.
