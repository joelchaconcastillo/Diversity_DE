In the last decade, DE has been recognized as one of the most promising EAs, likely for its eficient and simple approach to solve optimization problems.
%
Specifically, the DE variants have been highly present in several optimization competitions, principally in the Congress on Evolutionary Computation (CEC).
%
In fact DE occuped the top places in several optimization scenarios as are single-objective, multi-objective, contrained problems, large scale problems, dynamic problems, multi-niche landscape problemas and learned based problems.
%
In this work we are interested in the bias of designed of DE algorithms and CEC competitions problems througth the last years.
%

In CEC 2005 competition on real parameter optimization \cite{CEC2005}, on 10-D problems classical DE secured 2nd rank and a self-adaptive DE variant called SaDE secured third rank although they performed poorly over 30-D problems.
%
Later in CEC 2006 on constrained problems \cite{CEC2006} DE algorithms obtained first with $\epsilon$ constrained Rank-based Differential Evolution ($\epsilon$ RDE) and third place with SaDE.
%

Multi-objective optimization problems were proposed in CEC 2007 \cite{CEC2007} where DE obtained the second place with the based DE Generalized Differential Evolution 3 (GDE3), it is importat take into account that later in CEC 2009 the first place was reached by the Multi-Objective Evolutionary Algorithm Based in Decomposition (MOEA/D) which applies the DE operators instead of its old version (used genetic operators - SBX).

%
In the large scale global optimization (CEC 2008) \cite{CEC2008} a Self-adaptive DE (jDEdynNP-F) reached the third place, unfortunately int later competitions (CEC 2010) DE algorithms were not in the top rank, this could be an indicator of the weakness of DE in large scale problems \cite{segura2015improving}.

In CEC 2010 competition on constrained real-parameter optimization \cite{CEC2010} the first place was reached by the $\epsilon$ Constrained DE with gradient based mutation ($\epsilon$ Deg) and the third place by the Self-adaptive DE for solving constrained optimization (jDEsoco).

In CEC 2011 competition with real world optimization problems the second and third places were reached by Hybrid DE (DE-$\Lambda_{CR}$) and Self-adaptive Multi-Operator DE (SAMODE).
%
%
Later in CEC 2014 the first place was reached by the Linear Population Size Reduction Success-History Based Adaptive DE with Linear Population Size Reduction (L-SHAE) in the single objective real parameter optimization scenario.
%



Although a powerfull variant of ES, known as restart covariance matrix adaptation ES (CMA-ES) yielded better results than classical and self-adaptive DE, later on many improver DE variants were proposed in the period 2006-2017.
%
Hence, another rigorous comparison is needed to determine how well these variants might compete against the restart CMA-ES and many othe real parameter optimizers over the standard numerical benchmarks.
%
It is also interesting to note that the variants of DE continued to secure front ranks in the subsequent CEC competitions like CEC 2006 competition on constrained real parameter optimization (first rank), CEC 2007 competition on multi-objective optimization (second rank), CEC 2008 competition on large scale global optimization (third rank), CEC 2009 competition on multi-objective optimization (first rank was taken by a DE-based algorithm MOEA/D for unconstrained problems), and CEC 2009 competition on evolutionary computation in dynamic and uncertain enviroments (first rank).
%


It is important take into account that present three imporant categories principal DE variants: adaptive control parameters, Hybrid and Multi-Operator.
